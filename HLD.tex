%Copyright 2014 Jean-Philippe Eisenbarth
%This program is free software: you can 
%redistribute it and/or modify it under the terms of the GNU General Public 
%License as published by the Free Software Foundation, either version 3 of the 
%License, or (at your option) any later version.
%This program is distributed in the hope that it will be useful,but WITHOUT ANY 
%WARRANTY; without even the implied warranty of MERCHANTABILITY or FITNESS FOR A 
%PARTICULAR PURPOSE. See the GNU General Public License for more details.
%You should have received a copy of the GNU General Public License along with 
%this program.  If not, see <http://www.gnu.org/licenses/>.

%Based on the code of Yiannis Lazarides
%http://tex.stackexchange.com/questions/42602/software-requirements-specification-with-latex
%http://tex.stackexchange.com/users/963/yiannis-lazarides
%Also based on the template of Karl E. Wiegers
%http://www.se.rit.edu/~emad/teaching/slides/srs_template_sep14.pdf
%http://karlwiegers.com
\documentclass{scrreprt}
\usepackage{listings}
\usepackage{underscore}
\usepackage[bookmarks=true]{hyperref}
\usepackage[utf8]{inputenc}
\usepackage[english]{babel}
\usepackage{xcolor,colortbl}
\usepackage{makecell}
\usepackage{graphicx}
\hypersetup{
	bookmarks=false,    % show bookmarks bar?
	pdftitle={High-leve Design Document},    % title
	pdfauthor={Mohammad Arastu, Raphael Arzberger, Martin Fuhrmann},                     % author
	pdfsubject={TeX and LaTeX},                        % subject of the document
	pdfkeywords={TeX, LaTeX, graphics, images}, % list of keywords
	colorlinks=true,       % false: boxed links; true: colored links
	linkcolor=blue,       % color of internal links
	citecolor=black,       % color of links to bibliography
	filecolor=black,        % color of file links
	urlcolor=purple,        % color of external links
	linktoc=page            % only page is linked
}%
\newcolumntype{L}[1]{>{\raggedright\let\newline\\\arraybackslash\hspace{0pt}}m{#1}}
\newcolumntype{C}[1]{>{\centering\let\newline\\\arraybackslash\hspace{0pt}}m{#1}}
\newcolumntype{R}[1]{>{\raggedleft\let\newline\\\arraybackslash\hspace{0pt}}m{#1}}
\def\myversion{0.1}
\definecolor{LightCyan}{rgb}{0.88,1,1}
\date{}
%\title
\usepackage{hyperref}
\begin{document}
	
	\begin{flushright}
		\rule{16cm}{5pt}\vskip1cm
		\begin{bfseries}
			\Huge{High Level Design Document}\\
			\vspace{1.9cm}
			for\\
			\vspace{1.9cm}
			Timetable Planner\\
			\vspace{1.9cm}
			\LARGE{Version \myversion}\\
			\vspace{1.9cm}
			Prepared by \\ {\small Mohamad Arastu, Raphael Arzberger, Martin Fuhrmann}
			\vspace{1.9cm} \\
			\today\\
		\end{bfseries}
	\end{flushright}
	
	\tableofcontents
	
	
	\chapter*{Revision History}
	
	\begin{center}
		\begin{tabular}{|c|c|c|c|}
			\hline
			Date & Reason For Changes & Responsible Person & Version\\
			\hline
			06.12.2022 & project start & Mohammad Arastu, & \\&&Raphael Arzberger, & \\&&Martin Fuhrmann & 0.1\\
			\hline
		\end{tabular}
	\end{center}
	
	\chapter{Introduction}
	
	\section{Purpose}
	
	The FH++ app, which is being developed as part of this software project, is intended to be a useful tool for students at the FH Campus Wien and is meant to be an addition to existing services like FH Portal and Moodle. \\ \\
	The goal is to offer users of the app the opportunity to make their studies as efficient as possible.
	By using the interfaces of the FH portal and those of the open data portal of the City of Vienna, which enables access to data from Wiener Linien,
	the users should be empowered to plan their everyday studies in a time-optimizing manner.
	The arrival at the FH can be planned by combining information from the individual timetable and current traffic data for public transport. \\
	As an example, in case the metro U1 is used for getting to the FH, the app is intended to warn of a potentially later arrival time in the event of disruptions in the subway network.
	For navigation within the FH building, the user should be able to use a navigation aid. \\
	In addition, the app should also allow to read and send emails from the FH mail account.
	
	
\end{document}
